% [致谢]
\acknowledgments
\ifthenelse{\equal{\Show{}}{\show{}}}{% 是否显示[致谢]章
\begin{spacing}{1.3}\xiaosi %致谢内容为小四号
%%%%%%%%%%%%%%%%%%%%%%%%%%%%%%%%%%%%%%%%%%%%%%%%%%%%%%%%
% 内容由此开始

致谢中主要感谢指导教师和在学术方面对论文的完成有直接贡献及重要帮助的团体和人士,以及感谢给予转载和引用权的资料、图片、文献、研究思想和设想的所有者。致谢中还可以感谢提供研究经费及实验装置的基金会或企业等单位和人士。致谢辞应谦虚诚恳,实事求是,切记浮夸与庸俗之词。

\begin{enumerate}[label=\arabic*)]
	\item 致谢对象仅限对完成课题研究和论文写作过程给予指导和帮助的导师、任课教师、校内外专家、实验技术人员、同学等。
	\item 致谢内容以精练的叙述性文字内容为主,用词应含蓄、笼统、简朴,不宜出现感情色彩浓厚和流于俗套的溢美之词,不宜出现图表等。
\end{enumerate}

\par * 嗯,感谢完所有人之后,也请记得感谢一下自己 * \par
\vspace{18em}
\parbox[t][1cm][b]{\textwidth}{\hspace{20em}{
		签名:\par}}
	
\parbox[t][1cm][b]{\textwidth}{\hspace{23em}{
			\hspace{3em}年\hspace{2em}月\hspace{2em}日}}
		
% 内容到结束		
%%%%%%%%%%%%%%%%%%%%%%%%%%%%%%%%%%%%%%%%%%%%%%%%%%%%%%%%
\end{spacing}		
}{}