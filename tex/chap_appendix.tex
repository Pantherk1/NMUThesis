% [附录]
\appendix
\begin{spacing}{1.3}\xiaosi %附录内容为小四号
%%%%%%%%%%%%%%%%%%%%%%%%%%%%%%%%%%%%%%%%%%%%%%%%%%%%%%%%
% 内容由此开始

\begin{enumerate}[label=\arabic*)]
\item 主要列正文内容过于冗长的公式推导,供查读方便所需的辅助性数学工具或表格;重复性数据图表;论文使用缩写、程序全文及说明等。

\item 附录编号顺序依次为附录1、附录2、附录3……,每个附录应有标题。
\end{enumerate}
下列内容可以作为附录:

\begin{enumerate}[label=\arabic*)]
\item 为了整篇论文材料的完整,但编入正文又有损于编排的条理和逻辑性,这一材料包括比正文更为详尽的信息、研究方法和技术更深入的叙述,建议可以阅读的参考文献题录,对了解正文内容有用的补充信息等;
\item 由于篇幅过大或取材于复制品而不便于编入正文的材料;
\item 不便于编入正文的罕见的珍贵或需要特别保密的技术细节和详细方案(这中情况可单列成册);
\item 对一般读者并非必要阅读,但对专业同行有参考价值的资料;
\item 某些重要的原始数据、过长的数学推导、计算程序、框图、结构图、注释、统计表、计算机打印输出文件等。
\end{enumerate}

\par * 嗯,自由发挥吧 * \par

% 内容到此结束
%%%%%%%%%%%%%%%%%%%%%%%%%%%%%%%%%%%%%%%%%%%%%%%%%%%%%%%%
\end{spacing}
\ifthenelse{\equal{\@beginright}{off}}{\clearpage}{\clearautopage}