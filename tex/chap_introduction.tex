% [绪论]
% 此次为本LaTeX模板的简介
\chapter{绪论}
大家好,这是北方民族大学学位论文\LaTeX{}模板(\CTeX{}-Based)---\NMUThesis{}。

\NMUThesis{}为北民大研究生学位论文模板,适用于文史类和理工类论文(博士、学术硕士、专业硕士),支持单双面打印以及盲审版、最终版排版。本\LaTeX{}模板参考自院教字〔2003〕169号《北方民族大学研究生学位论文格式和要求》(以下简称《格式》),具体要求请参见《格式》,最终成文格式需参考学院要求及打印方意见。本模板中大量内容和说明直接摘抄自《格式》,基本覆盖了论文内容和格式方面的要求。

本模板改写自《北京航空航天大学学术论文\LaTeX{}模板》,部分样例参考《浙江大学研究生硕士(博士)学位论文\LaTeX{}模板》,代码已上传至GitHub\footnote{\href{https://github.com/WizenZhang/NMUThesis}{\LaTeX{}模板下载地址:https://github.com/WizenZhang/NMUThesis}}。文献著录BibTeX样式采用Haixing Hu开源的2005版参考文献著录BibTeX样式\href{https://github.com/Haixing-Hu/GBT7714-2005-BibTeX-Style}{GBT7714-2005}及Zeping Lee开源的2015版参考文献著录BibTeX样式\href{https://github.com/zepinglee/gbt7714-bibtex-style}{GBT7714-2015},在此感谢两位的开源分享。

另附\LaTeX{}入门视频教程\footnote{\href{https://blog.csdn.net/so_geili/article/details/51702564}{\LaTeX{}入门视频教程:https://blog.csdn.net/so\_geili/article/details/51702564}},请自行下载,更多例程参看第\ref{sec:sample}、\ref{sec:instruction}章。

意见及问题反馈请联系:\\
\indent E-mail:wizen\_zhang@163.com\\
\indent GitHub:\href{https://github.com/WizenZhang/NMUThesis/issues}{https://github.com/WizenZhang/NMUThesis/issues}

%%============================
\section{概述}
硕士研究生学位论文是学位申请人为申请硕士学位而撰写的学术论文,它集中表明了作者在研究工作中获得的新成果,是评判学位申请人学术水平的重要依据和获得学位的必要条件之一,也是科研领域中的重要文献资料和社会的宝贵财富。为提高我校硕士学位论文的质量,规范学位论文格式,特作如下规定。

%%============================
\section{基本要求}

\begin{enumerate}[label=\arabic*)]
	\item 硕士学位论文应能表明作者确已在本门学科上掌握了坚实的基础理论和系统的专门知识,并对所研究课题有新的见解,有从事科学研究工作或独立担负专门技术工作的能力。
	
	\item 除外语专业外,学位论文一般用中文撰写,硕士学位论文正文应不少于2万字。学位论文内容应立论正确、推理严谨、文字简练、层次分明、说理透彻、数据真实可靠。
	
	\item 量和单位及其符号均应符合国家标准的规定,国家标准中未规定的,应执行国际标准或行业标准;不同的量必须用不同的符号表示,不得一符多义,含义相同的量则必须用同一符号表示。学位论文应用最新颁布的汉语简化文字,符合《出版物汉字使用管理规定》;专业术语应统一使用全国自然科学名词审定委员会公布的各学科名词,或本学科权威和期刊通用的专业术语,且前后应一致;标点符号的使用应符合国家标准《标点符号用法》的规定;数字的使用应符合国家标准《出版物上数字用法的规定》。
	
	\item 图要精选,切忌与文字或表内容重复,图中文字、数据和符号应准确无误且与文字叙述一致,图应有图名,图名应简洁明确且与图中内容相符。表应用表序和表名,表名应简洁并与内容相符。图、表和公式应分别顺序编号。
\end{enumerate}

%%============================

\section{版式及其它要求}

%%============================

%%----------------------
\subsection{开本及版心}
{\bfseries 论文开本大小}:210mm$\times$297mm(标准A4纸)。

{\bfseries 论文版心}:左边距:30mm,右边距:25mm,上边距:30mm,下边距:25mm,页眉边距:23mm,页脚边距:20mm。
%%----------------------
\subsection{页眉及页脚}
\label{sec:error1}
\begin{enumerate}[label=\arabic*)]
	\item 从正文开始各页均加有页眉、页脚,文字均采用小五号宋体。
	
	\item 页眉左侧为“北方民族大学$\times$ $\times$ $\times$ 届硕士学位论文”,右侧为一级标题名称;页眉下横线为上粗下细文武线(3磅)。
	
	\item 页码格式为“-1-”,单面打印时,插入的页码排在页脚居中的位置;双面打印时,插入的页码分别排在页脚左右侧。
	
	\item 从内封面到目录,均用英文页码,如“I、II、III”,从引言到论文末页,页码用阿拉伯数字,如“-1- \highlight{、}-2-、-3-”。
\end{enumerate}

%%----------------------
\subsection{封面}

\begin{enumerate}[label=\arabic*)]
	\item 论文内外封面内容一样,外封皮用草绿色暗纹纸。
	
	\item 论文题目中英文对照,均可分两行排列;中文用黑体二号字,英文用Times New Roman三号字。
	
	\item 分类号按《中国图书资料分类法》要求查询填写。
	
	\item 密级:涉密论文,学院学位评定分委员会根据国家规定的密级范围和法定程序审查确定,并注明相应的保密年限;不需保密的应填写“公开”。
	
	\item 论文完成日期统一用阿拉伯数字填写。
\end{enumerate}

%%----------------------

%%----------------------
\subsection{独创性声明和使用授权书}

独创性声明和关于论文使用授权的说明附于内封面后,需由研究生和指导教师本人签字。

%%----------------------
%%============================

\section{论文各组成部分要求}

%%============================
%%----------------------
\subsection{摘要及关键词}

\begin{enumerate}[label=\arabic*)]
	\item 摘要即摘录论文要点,是论文要点不加注释和评论的一篇完整的陈述性短文,具有很强的自含性和独立性,能独立使用和被引用。
	
	\item 摘要应含有学位论文全文的主要信息,一般包括研究目的、研究方法、所取得的结果和结论。论文摘要应突出新见解或创新性。
	
	\item 摘要的详简度视论文的内容、性质而定,硕士学位论文摘要一般为500$\sim$600字,但不能超过1000字。
	
	\item 摘要中一般不用图、表、化学结构式、计算机程序,不用非公知公用的符号、术语和非法定的计量单位。
	
	\item “摘要” 居中用三号黑体字,3倍行间距;“关键词” 另起一行置于摘要下方,左对齐,用四号宋体加粗;摘要和关键词的内容用小四号宋体,行间距为1.5倍行距。
	
	\item 摘要一般为3至5个,中间以“,”分隔,涉及的内容、领域从大到小排列,便于文献编目与查询。
	
	\item 应有与中文摘要和关键词相对应的英文摘要和关键词。英文摘要用词要准确使用本学科通用词汇;摘要中主语(作者)常常省略,因而一般使用被动语态;应使用正确的时态并要注意主谓语的一致,必要的冠词不能省略。
	
	\item 英文摘要和关键词字号与中文一样,用Times New Roman字体;涉及到的姓名、书名等用斜体。
\end{enumerate}

%%----------------------
\subsection{目录}

\begin{enumerate}[label=\arabic*)]
	\item 目录依论文内的章节标题次序排列,标题应该简明扼要。
	
	\item 目录中仅出现两级标题,文史类目录标题为第一章、第一节,理工类目录标题为第一章、1.1。
	
	\item “目录” 居中用黑体二号字,一级标题左对齐用宋体四号字,二级标题与一级标题左空一个字的位置,用宋体小四号字。

\end{enumerate}

%%----------------------
\subsection{正文}

\begin{enumerate}[label=\arabic*)]
	\item 正文是论文的主体,一般由标题、文字叙述、图、表和公式等五个部分构成。写作形式可因科研项目的性质不同而变化,一般可包括理论分析、计算方法、实验装置和测试方法,经过整理加工的实验结果分析和讲座,与理论计算结果的比较以及本研究方法与已有研究方法的比较等。
	
	\item 正文分章节撰写,每章都另起一页。
	
	\item 正文内容使用五号宋体字,行间距为1.5倍行距。
	
\end{enumerate}

%%----------------------
\subsection{标题}
\begin{enumerate}[label=\arabic*)]
	\item 论文标题是以最恰当、最简明的词语反映论文中最重要的特定内容的逻辑组合。标题既要准确地描述内容,又要尽可能地短,一级标题一般不宜超过36个字。标题应该避免使用不常见的缩略词、字符、代号和公式等。
	
	\item 论文标题一般分为三级,文史类与理工类标题格式不同,具体如下:
	
	{\bfseries 文史类}:
	
	第一章(一级标题,居中,黑体三号字,3倍行间距)
	
	第一节(二级标题,居中,黑体四号字,2.5倍行间距)
	
	一、(三级标题,首行缩进2字符,黑体小四号字,2倍行间距)
	如有四五六级标题,可按如下格式:
	
	(一)(四级标题,首行缩进2字符,宋体五号字,2倍行间距)
	1.(五级标题,首行缩进2字符,宋体五号字,2倍行间距)
	
	(1)(六级标题,首行缩进2字符,宋体五号字,2倍行间距)
	
	{\bfseries 理工类}:
	
	第一章(一级标题,居中,黑体三号字,3倍行间距)
	
	1.1(二级标题,左对齐,黑体四号字,2.5倍行间距)
	
	1.1.1(三级标题,左对齐,黑体小四号字,2倍行间距)
	
	\item “参考文献”、“附录”、“致谢”、“个人简介”等标题为居中黑体三号字,3倍行间距;内容使用宋体小四号字,1.5倍行间距。
	
\end{enumerate}

%%----------------------
\subsection{注释}

\begin{enumerate}[label=\arabic*)]
	\item 所有引用、参考、借用的资料数据及他人成果必须标明出处,严禁抄袭、剽窃。
	
	\item 引用文献标注方式应全文统一,文中引用内容使用上标标注,以 \textcircled{1}、\textcircled{2}等为编号标于所引内容最末句右上角,用小五号宋体字;解释内容采用脚注\footnote{更多示例参看第\ref{sec:instruction}章}方式,以\textcircled{1}、\textcircled{2}为序号置于页下,用小五号宋体字,两端对齐,单倍行距。
	
	\item 不同页的脚注序号不需要连续编号;同一页几处引用同一文献时,将所有序号一起列出,只标注一次出处。
	
\end{enumerate}

%%----------------------
\subsection{参考文献}

\begin{enumerate}[label=\arabic*)]
	\item 参考文献采用尾注形式,标注于正文结束之后,不得罗列在各章节后。
	
	\item 各类文献资料的排列格式为:
	
	{\bfseries 期刊类}:
	[序号]作者.题目.刊名,出版年份,卷号(期号)
	
	{\bfseries 专(译)著类}:
	[序号]作者.书名(,译者).出版地:出版社,出版年,起止页码
	
	{\bfseries 论文集}:
	[序号]作者.题名,见(英文用In),主编,论文集名,出版地:出版社,出版年,起止页码
	
	{\bfseries 学位论文}:
	[序号]作者,题名,授予单位所在地:授予单位,授予年
	
	{\bfseries 专利}:
	[序号]申请者,专利名,国别,专利文献种类,专利号,出版日期
	
	{\bfseries 技术标准}:
	[序号]发布单位,标准代号,标准顺序号-发布年,标准名称,出版地,出版者,出版日期
	
	{\bfseries 电子文献}:
	[序号]作者.题名.获取或访问路径
	
\end{enumerate}

%%----------------------
\subsection{附录(非必要)}

\begin{enumerate}[label=\arabic*)]
	\item 主要列正文内容过于冗长的公式推导,供查读方便所需的辅助性数学工具或表格;重复性数据图表;论文使用缩写、程序全文及说明等。
	
	\item 附录编号顺序依次为附录1、附录2、附录3……,每个附录应有标题。
	
\end{enumerate}

%%----------------------
\subsection{致谢}

\begin{enumerate}[label=\arabic*)]
	\item 致谢对象仅限对完成课题研究和论文写作过程给予指导和帮助的导师、任课教师、校内外专家、实验技术人员、同学等。
	
	\item 致谢内容以精练的叙述性文字内容为主,用词应含蓄、笼统、简朴,不宜出现感情色彩浓厚和流于俗套的溢美之词,不宜出现图表等。
	
\end{enumerate}

%%----------------------
\subsection{个人简介}

\begin{enumerate}[label=\arabic*)]
	\item 简要介绍自己,内容包括姓名,性别,民族,籍贯,第一学历毕业院校及专业,取得的学位。
	
	\item 在研期间发表的论文,内容包括发表刊物名称,年月、卷册号,页码、论文作者排序及署名单位名称等,罗列论文以发表的时间先后排列。
	
\end{enumerate}

%%============================

\section{编排顺序及打印及装订等要求}

\begin{enumerate}[label=\arabic*)]
	\item 学位论文的编排顺序为外封面、内封面、独创性声明和授权说明、中文摘要、英文摘要、目录、引言/绪论、正文、结论/结语、注释和参考文献、附录、致谢、个人简介等部分。
	
	\item 学位论文内容一律用计算机编辑,用A4规格纸打印,按以上要求装订成册(不得用活页夹装订)。
\end{enumerate}

学位论文清单如表\ref{tab:papercomponents}。

\begin{table}[h]
	\caption{学位论文清单}
	\label{tab:papercomponents}
	\centering
	\begin{tabular}{cp{16\ccwd}p{4cm}}
		\toprule
		{\bfseries 装订顺序} & \multicolumn{1}{c} {\bfseries 内容} & \multicolumn{1}{c} {\bfseries 说明}  \\
		\midrule
		1 & 封面            & 盲审版和最终版 \\        
		3 & 独创性声明和使用授权书 & 盲审版论文无此项 \\
		4 & 中文摘要        & \\
		5 & 英文摘要        & \\
		6 & 目录            & 文史类和理工类\\
		7 & 正文            & 文史类和理工类\\
		8 & 结论/结语	        & \\
		9 & 参考文献        & \\
		10& 附录            & 非必要 \\
		12& 致谢            & 盲审版论文无此项 \\
		13& 个人简介        & 盲审版论文无此项 \\
		\bottomrule
	\end{tabular}
\end{table}
%%============================